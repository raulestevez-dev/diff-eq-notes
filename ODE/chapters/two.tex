\documentclass[../ode.tex]{subfiles}
\begin{document}
    \chapter{\sffamily Second-Order Linear Equations}
    
    In this chapter we study DE's of the type:
    \begin{equation*}
        ax'' + bx' + c = f(t)
    \end{equation*}
    
    This type of equation is of great use in chassical mechanics and electrical circuit, as they serve as prototypes for
    oscillationg systems, oscillation systems with damping and forced vibrations.

    \section{\sffamily Equations with constant coeficients}

    As with first-order ODE we have that:
    \begin{equation*}
        ax'' + bx' +cx = 0
    \end{equation*}
    is called an \emph{homogeneous linear equation with constant coeficients}. The \emph{homogeneous} part refers to the fact that
    the RHS is equal to zero.

    When we have also that:
    \begin{equation*}
        x(0) = x_0 \quad x'(0) = x_1
    \end{equation*}
    we are faced with an \emph{initial value problem} (IVP).
    
    \subsection{\sffamily Constant coeficients interpretation}
    Let's interpret the equation:
    \begin{equation*}
        mx'' + \gamma x' + kx =0 \iff LI'' + RI' + \frac{1}{C}I = 0
    \end{equation*}
    These are the equations for the damped spring and the LCR circuit. The constant coneficients: $m/L$,$\gamma/R$ and
    $k/\frac{1}{C}$ have a physical interpretation:
    \begin{itemize}
        \item  $m/L \longrightarrow$ It's the inertial term: in mechanics is the \emph{mass} and in LCR oscillator is the
            \emph{inductance} 
        \item  $\gamma / R\longrightarrow$ It's the dissipation or energy loss term: is analogous to the resistance or damping.
        \item  $k/ \frac{1}{C}\longrightarrow$ It's the energy storage term: is analogous to the (inverse) capacitance and the
            spring constant.
    \end{itemize}
 
    \subsection{\sffamily Solutions}
    
    \begin{thm}{Existence-Uniqueness}{SO-ex-un}
        The initial value problem given by:
        \begin{equation*}
            \begin{cases}
                ax''+bx'+cx = 0\\
                x(0) = x_0\\
                x'(0) = x_1
            \end{cases}
        \end{equation*}        
        has a unique solution that exist on $-\infty<t<\infty$. That means that an \emph{homogeneous} linear IVP has a unique
        solution for all reals.
    \end{thm}

    But, how do we find such solutions? This type of equations have a solution of the form:
    \begin{equation*}
        x(t) =c_1 x_1(t) + c_2 x_2(t)
    \end{equation*}
    Where $x_1(t)$ and $x_1(t)$ are independt (not multiple of each other) solutions.

    We can spect the solution to be of the form $x(t)=\mathrm{e}^{\lambda t} $ because the
    terms in the equation involve the derivatives of the solution multiplied by a constant. If we substitute $\mathrm{e}^{\lambda
    t} $ in the general equation we get:
    \begin{equation*}
        a\lambda^{2} + b\lambda + c =0 
    \end{equation*}
    that equation is called the \emph{characteristic equation}, and it's solutions \emph{eigenvalues}. Each value of $\lambda$
    gives a solution $\mathrm{e}^{\lambda t} $ to the equation. As any second-order equation it can have real or complex conjugate
    solutions, let's see what happends with the complex solutions.

    \subsection{\sffamily Complex eigenvalues}

    \begin{thm}{Complex eigenvalues}{complex-eigenvalues}
        If $x(t) = g(t) + j h(t)$ is a complex-valued solution of a second-order LDE then it's real and imaginary parts ($g(t)$
        and $h(t)$ respectively) are real valued solutions.
    \end{thm}
     
    So, let's study the solution $x(t)=\mathrm{e}^{\lambda t}=\mathrm{e}^{(\alpha + j\beta) t}$, using Euler's formula:
    \begin{equation*}
        \mathrm{e}^{(\alpha + j\beta) t} = \mathrm{e}^{\alpha t} \mathrm{e}^{j\beta t} = \mathrm{e}^{\alpha t} (\cos(\beta t) + j
        \sin(\beta t)) = \mathrm{e}^{\alpha t }\cos(\beta t) + j \mathrm{e}^{\alpha t} \sin(\beta t)
    \end{equation*}
    So, using the previous theorem we can conclude that:
    \begin{equation*}
        x_1(t) = \mathrm{e}^{\alpha t} \cos(\beta t) \quad x_2(t) = \mathrm{e}^{\alpha t} \sin(\beta t)
    \end{equation*}
    are both \emph{real-valued} solutions to the equation. The other comples root can be proven to give the same solutions (only
    difering in a sign, but they are the same \emph{independent} solutions).
    
    

    \subsection{\sffamily Resonance}
    
    The phenomenon of \emph{resonance} is a key element of vibrating systems. It occurst when the \emph{frequency} of a forcing
    term has the same frequency as the natural oscillations in the system. It gives rise to large amplitude oscillations. In the
    book they only give examples, boring to write in \LaTeX.
\end{document}
