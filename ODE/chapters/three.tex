\documentclass[../ode.tex]{subfiles}
\begin{document}
    \chapter{\sffamily Linear Systems}
    Let's start stating the any second-order linear differential equation is equivalent to a two unknow function first-order
    linear system, i.e.:
    \begin{equation*}
        ax'' + bx' + cx = 0 \iff 
        \begin{cases}
            y=x'\\
            ay' + by +cx =0
        \end{cases}  
    \end{equation*}
    
    The general form of a linear system is: 

    \begin{equation*}
        \begin{cases}
            x'=ax+by\\
            y'=cx+dy
        \end{cases}
    \end{equation*}
    

    \section{\sffamily Insight from the equations}
    
    As for the first-order linear equations, we can get some visual insight by inspection/ploting without even
    solving them. Similarly to the slope fields, we can plot the so called \emph{vector field} of the
    equation just by ploting the function:

    \begin{equation*}
        \vec{F}(x,y) =\begin{bmatrix} x'\\y' \end{bmatrix}  =\begin{bmatrix} ax+by\\cx+dy \end{bmatrix}\\
    \end{equation*}
    
    This plot gives the tangent vector of the solution in each point of the $xy$ plane. There are two special lines
    where the tangent vector is either vertial or horizontal, they are called the \emph{nullclines}. The \emph{$x$-nullcline} is
    the line $ax+by=0$ i.e. where $x'=0$ so the vector field is vertical. IDEM with the \emph{$y$-nullcline}. Note:
    \emph{null-cline} comes from \emph{in-cline} but \emph{null} because there zero (\emph{null}) incline in one component? Maybe.
    
    Now let's assume that we know the solution to the linear system i.e. we have the equations $x(t)$ and $y(t)$. We can plot the
    solution in two ways: as a time series or as a parametric plot. The former is the simplest and can give information about
    periocity, they tell us how both the states vary with time. But the latter is the most insteresting because we can think of
    the solutions as a parametric curve in the $xy$ plane i.e. we plot $\vv{x}(t) = (x(t), y(t))$ and we get an \emph{orbit}. The
    evolution in time of the orbit can be illustrated by also plotting the vector field for that solution.

    In some ocasions we could alse plot the solution as a \emph{one function} i.e. if we can eliminate the time parameter we can
    plot the obtained equation as a normal function $y(x)$.
            
    \section{\sffamily The eigenvalue problem}
    
    We can see that a linear system can be expressed in matrix form as:
    \begin{equation*}
        \begin{cases}
            x' = ax + by\\
            y' = cx + dy
        \end{cases}
        \iff 
        \begin{bmatrix} a & b \\ c & d \end{bmatrix} \begin{bmatrix} x \\ y  \end{bmatrix} = \begin{bmatrix} x' \\ y' \end{bmatrix} 
        \iff
        A\vv{x} = \vv{x}' 
    \end{equation*}
    
    As the derivative of the solution involves multiples of the solution function, we can attemp to find one of the form:
    \begin{equation*}
        x(t) = \vv{v} \mathrm{e}^{\lambda t} 
    \end{equation*}
    Where $\vv{v} $ and $\lambda$ are constant. Substituting this and its derivative $x'(t)=\lambda \vv{v} \mathrm{e}^{\lambda t} $
    in the equation in matrix form gives:
    \begin{gather*}
        A\vv{x} = \vv{x}' \implies A\vv{v} \mathrm{e}^{\lambda t} = \lambda \vv{v} \mathrm{e}^{\lambda t} \iff A\vv{v} = \lambda
        \vv{v} 
    \end{gather*}
      
    And this is a classic \emph{eigenvalue problem}. And if we can find the \emph{eigenpair} $(\vv{v}, \lambda)$ we have the
    solution to the linear system as $\vv{x} = \vv{v} \mathrm{e}^{\lambda t} $. We have transformed the problem of solving a
    differential equation to one of solving an algebraic problem.
    
\end{document}
