\documentclass[../pde.tex]{subfiles}
\begin{document}
    \chapter{\sffamily Introduction to Partial Differential Equations}
    
    \section{\sffamily Notation}
    We start by defining the typical functio that we'll be solving for in PDE's, in this case:
    \begin{align*}
        u : \mathbb{R}^2 &\rightarrow \mathbb{R} \\
            (x,t) &\mapsto u=u(x,t) 
    \end{align*}
    Here we have a function that maps \emph{2-tuples} (in this case position and time) to real numbers. The elements of the
    \emph{domain}, $x$ and $t $,  are called \emph{independent} variables and the element of the \emph{codomain}, $u$ is called
    the \emph{dependent} variable. This notation can be a little confusing because we are calling the function and the element of
    the codomain with the same name, but it's the more used notation and after adaptation is very clear. Note that in this
    example is a function from $\mathbb{R} ^2$ to $\mathbb{R} $ but it can have whatever domain it needs.
    
    Once we've defined the function we are solving for, we'll define the \emph{partial derivative with respect to $t$} of this function as:
    \begin{align*}
        u_{t}: \mathbb{R} ^2 & \rightarrow \mathbb{R} \\
        (x,t) &\mapsto u_{t}=(\partial_{t} u)(x,t)
    \end{align*}
    
    Here the \emph{independent} variables are still $x$ and $t$, and the \emph{dependent} variable is $u_{t}$ (the same as the
    function name). In this case we also use the same name for the function and the element of the codomain. Note the use of the $\partial_{t} $
    \emph{operator}. 
    \begin{itemize}
        \item $\partial_{t} u$ : is the partial derivate of $u$ w.r.t. $t$. It is a \emph{function} not a \emph{number}.
        \item $(\partial_{t} u)(x,t)$ : is the partial derivative of $u$ w.r.t. $t$ at the space-time point $(x,t)$. It is a
            \emph{number}. 
    \end{itemize}
    It's fundamental to undestand the difference between this two statements.
    
    For repeated differentiation we use the notation: 
    \begin{equation*}
        u_{tt}, u_{tx}, u_{txx}, \dots
    \end{equation*}
    and using the differential operator:
    \begin{equation*}
        \partial_{t}^2 u, \partial_{tx}u, \partial_{txx}u \dots
    \end{equation*}
    
    
   \subsection{\sffamily Kinds of PDE's}
   First we define a \emph{linear equation} as:
   \begin{equation*}
       Au_{xx}+Bu_{xy}+Cu_{yy}+Du_{x}+Eu_{y}+Fu=G
   \end{equation*}
   Where $A,B,C,D,E,F,G$ are all \emph{constants} or \emph{functions} of the independet variables ($x$ and $y$ ).

   For linear equation the are three basic types:
   \begin{itemize}
        \item \emph{Parabolic}: usually describe heat flow and diffusion processes. They satisfy the property $B^2-4AC=0$.
        \item \emph{Hyperbolic}: usually describe vibrating systems and wave motion. They satisfy the property $B^2-4AC > 0$.
         \item \emph{Elliptic}: usually describe steady-state phenomena. They satisfy the property $B^2-4AC<0$.
   \end{itemize}
   
    
\end{document}
