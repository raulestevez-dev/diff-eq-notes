\documentclass[../pde.tex]{subfiles}
\begin{document}
    \chapter{\sffamily Diffusion-Type Problems}
    First lets start by interpreting one of the most simple PDE's, the one dimensional \emph{heath equation}.
    \section{\sffamily Interpretation of the heat equation}
    First let's state the problem that we want to solve. Imagine a straight bar of metal. The position in the bar is modeled by
    the independet variable $x$ and time with $t$. The function $u(x,t)$ models the distribution of temperature in the bar at the
    point $x$ and time $t$. We can see this function in two forms. As a surface in the $xt$ plane such that for every time $t_0$ 
    we have a function $u(x)$ that represents the distribution of temperatures in the bar. Or we can conceptually think of
    $u(x,t)$ as a animation of a a function $u(x)$ that changes form as the animation advances.

    \begin{figure}[ht]
        \centering
        \incfig[1]{xt-heath}
        \caption{Visualization of $u(x,t)$ as a surface in the $xt$ plane. We can think of the function evolving in time as we
        travel through the $t$ line.}
        \label{fig:xt-heath}
    \end{figure}

    The one dimensional heath equation is a \emph{parabolic} equation of the form:
    \begin{equation*}
        u_{t}=\alpha^2 u_{xx}; \quad 0<x<L,\quad  0<t<\infty 
    \end{equation*}
    
    which relates the quantites:
    \begin{itemize}
        \item $u_{t} = $ the rate of change of the temperature of a point as time advances.
        \item $u_{xx} = $ the \emph{concavity} of the temperature distribution at the point $(x,t)$. 
    \end{itemize}
    
    What this equation tries to say is that the temperature $u(x,t)$ will increase or decrease accoding to whether $u_{xx}$ 
    is positive or negative.

    \subsection{\sffamily Interpretation of $\mathbf{u_{xx}}$}
    To interpret $u_{xx}$ lets steer to the numerical methods. Imagine that we have a discreet rod of metal (or a discreet number
    of thermocouples measuring the temperature of a continuos rod) with a temperature distribution $u(x_n,t)$. We can model second
    derivatives as a  (central) \emph{finite diference}:
    \begin{equation}\label{num-2-deriv}
        u_{xx}(x_n,t) = \frac{1}{h^2}[u(x_{n+1},t) - 2u(x_n,t) + u(x_{n-1},t)]
    \end{equation}
    We can rearrange that to give:
    \begin{equation*}        u_{xx}(x_n,t) = \frac{2}{h^2}\left[\frac{u(x_{n+1},t)+u(x_{n-1},t)}{2}-u(x_n,t)\right]
    \end{equation*}
    
    This lets us study the sign of $u_{xx}$ as the difference between the \emph{average} of the temperature arround the current point
    and the temperature of the current point.
    \begin{itemize}
        \item If the \emph{average is larger} than the temperature of the current point, then $u_{xx}$ will be positive. 
            So the temperture of the current point will increase with
            time (as $u_{xx} > 0 \implies u _{t}>0$ with the heath equation).
        \item If the \emph{average is lower} than the temperature of the point then the reverse happens: $u_{xx} < 0 $ and the 
            temperature of the point tends to decrease.
    \end{itemize}

    If we keep rearranging \eqref{num-2-deriv} we arrive to:

    \begin{align*}
        u_{xx} &= \frac{1}{h^2} \left[ u(x_{n+1},t) - u(x_{n},y) + u(x_{n-1},t) - u(x_{n},t)  \right] \\
               &= \frac{1}{h^2} \left[ \Delta u - \nabla u \right] 
    \end{align*}
        
    Where $\Delta$ and $\nabla$ are the forward and backward difference operators, respectivly.

    We can interpret this as the rate of change of the rate of change i.e. $\Delta u$ tells us how the temperature changes
    relative to the next point, and $\nabla u$ tells us how the temperature changes relative to the previous point. So the
    difference between this rates of change will tell us how the rate of change ($u_x$) changes as we move in the bar from left to
    right.\\
    If this double rate of change is positive then the function at that point is concave up 
    (like $\cup$) and so the temperature will tend to rise to equalice with it's vecinity. The reverse is also true, if the
    difference is negative it means that the function is concave down (like $\cap$ ) so the temperature will tend to decrease,
    givin it's energy to it's neighbors. The \emph{sign} of $u_{xx}$ tells the direction of the flux of energy i.e. if the point
    is gaining energy: $u_{xx}>0$; or if it's losing it: $u_{xx}<0$.

    \begin{figure}[ht]
        \centering
        \incfig[1]{uxx}
        \caption{This figure illustrates the tendency of the temperature at the points where the function is concave up and down.}
        \label{fig:uxx}
    \end{figure}

    \section{\sffamily Boundary and Initial conditions}
    
    Following with the example of the metal rod, we need to state what happens at the boundary of such rod. Does it lose
    temperature from its ends? Does it maintain a fixed temperature? This statements are called the \emph{boundary conditions},
    they say how our model behaves at its domain baoundary.\\
    All systems must start from some value of time (usually $t=0$, we can think of it as the time at which we begin to measure)
    and so we need to declare the state of the system at that point, this is called the \emph{initial condition}.

    If we have a boundary condition and a initial condition we have what is called an \emph{initial-boundary-value problem
    (IBVP)}.
    In the rod example we can have, let say, a fixed temperature at the ends of $T_1$ and $T_2$ respectively and a initial
    temperature of $T_0$ in all of rod. So the full problem would be:
    
    \begin{align*}
        &\text{PDE} \quad u_{t} = \alpha^2 u_{xx} \quad 0<x<L \quad 0<t<\infty \\[1em]
        &\text{BCs} \quad 
        \begin{cases}
            u(0,t) = T_1\\
            u(L,t) = T_2
        \end{cases} 
        0< t<\infty \\[1em]
        &\text{IC} \quad u(x,0) = T_0 \quad 0\le x \le L
    \end{align*}


    \section{\sffamily Boundary conditions to diffusion type problems}
    
    There are two main types of boundary conditions for diffusion type problems: Temperature of the boundaries or flux specified. 
    \subsection{\sffamily Temperature of the boundaries}
    It's what we did in the rod example, we mantained a constant temperature in the ends of the rod (the boundary). In general the
    temperature at the boundary can be a function of time. It can even be the case that we want to find the boundary
    temperatures that will force the temperature of the object to act in a suitable maner. 

    So we can force the ends of the rod to be at a temperature given by the functions $g_{1}(t)$ and $g_{2}(t)$.

    \subsection{\sffamily Flux specified}
    
    \subsection{\sffamily Definition of Flux}
    There exist two definitions for flux, one used in transport problems (heat transfer, mass transfer, fluid dynamics...) and
    another used in electromagnetism. Here we are using the transport definition.

    Flux is defined as \emph{the rate of flow of a propertie per unit area}, so it has dimenstions of \emph{[quantity]}$\cdot$
    \emph{[time]$^{-1}$} $\cdot$ \emph{[meter] $^{-2}$}. For example, the flow of a river is the amount of \emph{liters} that
    the river passes in one \emph{second}. And its \emph{flux} is then the \emph{flow} of the river per unit area of the cross
    section of the river. 

    \subsection{\sffamily Fourier's Law}
     \begin{figure}[ht]
        \centering
        \incfig[1]{heat-flow}
        \caption{Direction of heat flow in two points. When the derivative is $<0$ when interprete that the flux is positive i.e.
        energy is entering the point to raise it's temperature and reach an equilibrium.}
        \label{fig:heat-flow}
    \end{figure}
    
    One expression for the \emph{heat flux} is given by \emph{Fourier's Law} in two dimensions:
    \begin{equation}\label{furier-law}
        \phi_{q} = -\mathrm{k} \partial_{x} u
    \end{equation}
    And as a flux $\phi _{q}$ has units of $\mathrm{\frac{J}{s \cdot m}}$ or what is the same $\mathrm{\frac{W}{m}}$ (in the three
    dimensional case it has units with square meters, not meters). Here $\mathrm{k}$ is the \emph{thermal conductivity}, the measure
    of the ability of the material to conduct heat, with units of $\mathrm{\frac{W}{K}}$ (In the three dimensional case:
    $\mathrm{\frac{W}{m\cdot K}}$).
    
    The interpretation is that where the temperature if falling ($u_{x}<0$ ) heath will enter the zone ($\phi_{q} > 0)$  to reach 
    equilibrium.

   
    
    \subsection{\sffamily Newton's law of cooling}
    The discreet analog of the \emph{Fourier's Law}  is the \emph{Newton's law of cooling}. This law can be derived from the
    Fourier's law by seting the temperature function as the sum of two coliding heavyside functions with diferent temperatures and
    solving the equations (i think?). In this version it states that the \emph{heat flux} between two objects with temperatures
    $T_{1}$ and $T_{2}$ is:
    \begin{equation}\label{newton-cooling}
        \phi_{q} = h(T_{2}-T_{1})
    \end{equation}
    That means that the \emph{heat flux}  between two objects is proportional to the temperature difference between the two.
    Usually one of the objects is the ambient (hence the name of \emph{cooling law}).

    Here $\mathrm{h} $ is the \emph{heat transfer coeficient} with units of $\mathrm{\frac{W}{m^2 \cdot K}} $ in three dimensions.
    
    
    

\end{document}
