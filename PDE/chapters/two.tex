\documentclass[../pde.tex]{subfiles}
\begin{document}
    \chapter{\sffamily Diffusion-Type Problems}
    First lets start by interpreting one of the most simple PDE's, the one dimensional \emph{heath equation}.
    \subsection{\sffamily Interpretation of the heat equation}
    First let's state the problem that we want to solve. Imagine a straight bar of metal. The position in the bar is modeled by
    the independet variable $x$ and time with $t$. The function $u(x,t)$ models the distribution of temperature in the bar at the
    point $x$ and time $t$. We can see this function in two forms. As a surface in the $xt$ plane such that for every time $t_0$ 
    we have a function $u(x)$ that represents the distribution of temperatures in the bar. Or we can conceptually thing of
    $u(x,t)$ as a animation of a a function $u(x)$ that changes form as the animation advances.

    \begin{figure}[ht]
        \centering
        \incfig[1]{xt-heath}
        \caption{Visualization of $u(x,t)$ as a surface in the $xt$ plane. We can think of the function evolving in time as we
        travel through the $t$ line.}
        \label{fig:xt-heath}
    \end{figure}

    The one dimensional heath equation is a \emph{parabolic} equation of the form:
    \begin{equation*}
        u_{t}=\alpha^2 u_{xx}; \quad 0<x<L,\quad  0<t<\infty 
    \end{equation*}
    
    which relates the quantites:
    \begin{itemize}
        \item $u_{t} = $ the rate of change of the temperature of a point as time advances.
        \item $u_{xx} = $ the \emph{concavity} of the temperature distribution at the point $(x,t)$. 
    \end{itemize}
    
    What this equation tries to say is that the temperature $u(x,t)$ will increase or decrease accoding to whether $u_{xx}$ 
    is positive or negative.

    \subsection{\sffamily Interpretation of $\mathbf{u_{xx}}$}
    To interpret $u_{xx}$ lets steer to the numerical methods. Imagine that we have a discreet rod of metal (or a discreet number
    of thermocouples measuring the temperature of a continuos rod) with a temperature distribution $u(x_n,t)$. We can model second
    derivatives as a  (central) \emph{finite diference}:
    \begin{equation*}
        u_{xx}(x_n,t) = \frac{1}{h^2}[u(x_{n+1},t) - 2u(x_n,t) + u(x_{n-1},t)]
    \end{equation*}
    We can rearrange that to give:
    \begin{equation}\label{num-2-deriv}
        u_{xx}(x_n,t) = \frac{2}{h^2}\left[\frac{u(x_{n+1},t)+u(x_{n-1},t)}{2}-u(x_n,t)\right]
    \end{equation}
    
    This lets us study the sign of $u_{xx}$ as the difference between the \emph{average} of the temperature arround the current point
    and the temperature of the current point.
    \begin{itemize}
        \item If the \emph{average is larger} than the temperature of the current point, then $u_{xx}$ will be positive. 
            So the temperture of the current point will increase with
            time (as $u_{xx} > 0 \implies u _{t}>0$ with the heath equation).
        \item If the \emph{average is lower} than the temperature of the point then the reverse happens: $u_{xx} < 0 $ and the 
            temperature of the point tends to decrease.
    \end{itemize}

    If we keep rearranging \eqref{num-2-deriv} we arrive to:

    \begin{align*}
        u_{xx} &= \frac{1}{h^2} \left[ u(x_{n+1},t) - u(x_{n},y) + u(x_{n-1},t) - u(x_{n},t)  \right] \\
               &= \frac{1}{h^2} \left[ \Delta u - \nabla u \right] 
    \end{align*}
        
    Where $\Delta$ and $\nabla$ are the forward and backward difference operators, respectivly.

    We can interpret this as the rate of change of the rate of change i.e. $\Delta u$ tells us how the temperature changes
    relative to the next point, and $\nabla u$ tells us how the temperature changes relative to the previous point. So the
    difference between this rates of change will tell us how the rate of change ($u_x$) changes.\\
    If it is positive then the function at that point is concave up 
    (like $\cup$) and so the temperature will tend to rise to equalice with it's vecinity. The reverse is also true, if the
    difference is negative it means that the function is concave down (like $\cap$ ) so the temperature will tend to decrease,
    givin it's energy to it's neighbors. The \emph{sign} of $u_{xx}$ tells the direction of the flux of energy i.e. if the point
    is gaining energy: $u_{xx}>0$; or if it's losing it: $u_{xx}<0$.

    \begin{figure}[ht]
        \centering
        \incfig[1]{uxx}
        \caption{This figure illustrates the tendency of the temperature at the points where the function is concave up and down.}
        \label{fig:uxx}
    \end{figure}
    
    

\end{document}
